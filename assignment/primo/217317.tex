\documentclass{report}

\usepackage[margin=1.5cm]{geometry}
\usepackage{hyperref}

\author{Blascovich Alessio\\ 217317\\ \href{mailto:alessio.blascovich@studenti.unitn.it}{alessio.blascovich@studenti.unitn.it}}
\title{Sistemi informativi\\ Primo assignement}
\date{}

\begin{document}
   \maketitle

   \begin{abstract}
        Analizzare i ruoli del personale del supermercato SOTTOCASA (200 punti vendita sparsi in tutta Italia, di cui 10 ipermercati), classificarli in base alla piramide di Anthony e provare a tratteggiarne l'esigenza informativa.
   \end{abstract}
   
   \chapter*{Personale esecutivo}
   Considerando le grandi dimensioni dell'azienda possiamo assumere possieda dei magazzini di proprietà e quindi abbia dovuto assumere:
   \begin{itemize}
        \item Addetto alla movimentazione delle merci nel magazzino e da/verso l'esterno.
        \begin{itemize}
            \item Questo tipo di figura avrà a disposizione una serie di strumenti per monitorare la merce (lettori di codici a barre, lettori RFID) e dovrà avere inoltre una serie di indicazioni rispetto a che ordini arriveranno, che ordini bisogna preparare per la spedizione e se bisogna spostare degli articoli all'interno del magazino.
        \end{itemize}

        \item Per portare la merce dal magazzino al supermercato la ditta sarà fornita di una flotta di autisti.
        \begin{itemize}
            \item Ad ogni autista dovrà essere fornita la lista dei viaggi da compiere nel suo turno e per ogni viaggio dovranno essere forniti i documenti di trasporto adeguati.
        \end{itemize}
   \end{itemize}
   Passando poi all'interno del punto vendita in sè possiamo trovare:
   \begin{itemize}
        \item Ogni punti vendita (dipendentemente dalle dimensioni) dovrà avere una squandra di magazzinieri che raccoglie gli ordini che arrivano dall'esterno.
        \begin{itemize}
            \item Ogni magazziniere dovrà essere dotato di scanner o lettore RFID per registrare la merce in entrata e segnare gli ordini arrivati nei docuemnti che gli sono stati assegnati ad inizio turno.
        \end{itemize}

        \item Ogni super/iper mercato dovrà avere una squadra di commessi.
        \begin{itemize}
            \item Ogni commesso dovrà essere costantemente notificato da un palmare dalla presenza/assenza di un articolo.\\
                Il palmare dovrà registrare, in un sistema interno al punto vendita, quando vengono aggiunti sullo scaffale $x$ unità di un prodotto e la cassa dovrà segnare quando ne vengono sotratte $y$.\\
                Il palmare dovrà anche notificare il magazzino quando un impegato preleva dal magazzino mentre la cassa dovrà notificare il sistema degli scaffali.
        \end{itemize}

        \item Ogni punto vendita dovrà possedere almeno un cassiere.
        \begin{itemize}
            \item Ogni cassa sarà provvista di lettore di codice a barre per registrare l'uscita di ogni tipo di articoli, la cassa viene sbloccata con il codice dell'operatore che permette di conoscere quanti introiti produce un cassiere in $n$ tempo.
        \end{itemize}

        \item Soprattutto gli ipermercati devono possedere dei banchi specifici salumi/pesce/pane con degli addetti.
        \begin{itemize}
            \item Quasi ogni giorno l'addetto deve ricevere i prezzi aggiornati delle merci e le bilancie che usa sono aggiornate di conseguenza da un sistema centralizzato.
        \end{itemize}
   \end{itemize}

   \chapter*{Direzione funzioanale}
   Come nel capitolo precedente parto dall'analizzare la struttura dei magazzini distrettuali.
   \begin{itemize}
        \item Ogni magazzino avrà bisogno di un direttore.
        \begin{itemize}
            \item I direttori dei magazzini dovranno leggere periodicamente i report delle movimentazioni degli articoli effettuando, di conseguenza, ordini dai grossisti ai quali la catena si appoggia.
        \end{itemize}

        \item Direttore di un punto vendita, presente in ogni supermercato.
        \begin{itemize}
            \item Il direttore dovrà monitorare lo stato del magazzino della filiale per effettuare gli ordini in un tempo ragionevole, emanando ordini al magazzino e docuementi generati in automatico per i magazzinieri.\\
            Dovrà leggendo i report economici del punto vendita inoltrare una richiesta al direttore di distretto per poter alterare l'organico del negozio.\\
            Infine leggendo in modo più specifico le categorie di prodotti venduti dovrà inoltrare all'alta direzione un report per poter decidere future promozioni.
        \end{itemize}

        \item Il direttore di distretto sovraintende a diversi punti vendita in un'area geografica.
        \begin{itemize}
            \item Leggendo i report econommici può decidere di alterrare l'organico di un punto vendita o di un magazzino.
            \item Decide alcuni fornitori locali di prodotti del luogo, tipicamente in Italia i vini cambiano di molto tra le varie zone geografiche.
            \item Producono un report di anadamento economico per ogni zona.
        \end{itemize}
   \end{itemize}

   \chapter*{Alta direzione}
   Presuppongo che l'alta direzione sia formata da più figure specializzate ma che chiameremo consiglio di amministrazione.\\
   Il consiglio di amministrazione può:
   \begin{itemize}
        \item Decidere l'apertura e chiusura di nuove sedi in base ai report economici di zona forniti dai direttori distrettuali.\\
            Solitamente le aperture e le chiusure devono essere mirate in un luogo dove la situazione è favorevole al cambiamento.
        \item In base ai prezzi forniti dai grossisti decidono da chi andarsi a fornire e di conseguenza quali saranno i prezzi al cliente.
        \item Dai report economici forniti dai direttori delle filiali e dai direttori dei magazzini possono decidere di togliere prodotti dagli scaffali, aggiungerne oppure iniziare delle promozioni.
   \end{itemize}
\end{document}