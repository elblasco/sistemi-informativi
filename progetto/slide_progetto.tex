\documentclass{beamer}
\usepackage{hyperref}
\newcommand{\credit}[1]{\par\hfill \footnotesize Fonte:~\itshape#1}

\usetheme{AnnArbor}
\usecolortheme{orchid}
\title{ERP: L'area logistica}
\author{Blascovich Alessio, Piero Fontanive}
\date{}

\begin{document}
    \begin{frame}
        \titlepage
    \end{frame}

    \section{Obbiettivi}
    \begin{frame}{Obbiettivi}
        La parte di sistema informativo rivolto alla logistica si occupa di:
        \begin{itemize}
            \item Tenere traccia del movimento della merce.\\
                \textit{Fornisce anche API per tracciare il pacco dall'esterno come i vari corrieri espresso.}
            \item Fornire dati analitici sulla merca.\\
                \textit{E' possibile fare un resoconto sulla disponibilità è la giacenza degli articoli.}
            \item Effettuare previsioni sullo stato dell'inventario.\\
                \textit{Dopo il black friday avrò bisogno di ordinare x unità di articolo 1 e y unità di articolo 2.}
        \end{itemize}
    \end{frame}

    \subsection{Evoluzione obbiettivi}
    \begin{frame}{Evoluzione obbiettivi}
        Nei sistemi più grandi ed evoluti è possibile compiere funzioni per:
        \begin{itemize}
            \item Localizzare a livello fisico l'ubicazione di un articolo.
            \item Tracciare origine e destinazione di un lotto di articoli o di un singola tipologia di articoli usandone la marticola.
            \item Muovere parzialmente o totalmente la merce in automatico.\\
                ome viene fatto nei magazzini di alcune multinazionali, un esempio può essere \href{https://www.youtube.com/watch?v=YL9XjyXsKKk}{questo}.
        \end{itemize}
    \end{frame}

    \section{Strutture di base}
    \begin{frame}{Strutture di base}
        La logistica si avvale di tre strutture base:
        \begin{enumerate}
            \item L'anagrafica degli articoli, ovvero la descrizione dei prodotti che un'azienda gestisce.
            \item La compisizione fisica e logica del magazzino dove si andrà ad operare.
            \item La movimentazione degli articoli, ovvero la rappresentazione dei movimenti compiuti.
        \end{enumerate}
    \end{frame}

    \subsection{Nominazione articoli}
    \begin{frame}{Intro -  Nominazione articoli}
        Un problema fondamentale è la standardizzazione nella nomenclatura della merce all'interno di un azienda.\\
        E' necessario trovare metodi di nomenclatura che creaino meno \textbf{omocodia} possibile e che siano facilmente leggibili.\\
        \vspace{1.5em}
        \textbf{E.g.}\\
        Una numerazione progressiva è facile da implementare ma crea difficoltà nella correlazione numero -$>$ prodotto.\\
        Il codice fiscale crea omocodie, nel 2015 erano presenti 35800 casi di persone vive con codici uguali.
        \credit{\href{https://www.agenziaentrate.gov.it/portale/documents/20143/318757/Audizione+del+Direttore+dell+Agenzia+delle+Entrate+10+02+2016_Audizione+-+Codice+Fiscale+e+Omocodie+-+10+feb+2016+df.pdf/202e5395-f622-62e5-219a-7ce894d538a4}{Agenzia delle entrate}} 
    \end{frame}
\end{document}